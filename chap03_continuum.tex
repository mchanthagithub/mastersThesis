%% This is an example first chapter.  You should put chapter/appendix that you
%% write into a separate file, and add a line \include{yourfilename} to
%% main.tex, where `yourfilename.tex' is the name of the chapter/appendix file.
%% You can process specific files by typing their names in at the 
%% \files=
%% prompt when you run the file main.tex through LaTeX.
\def\b_p{\bm{B}^{e,p}}
\chapter{Continuum Model}
We use two continuum models in this study. The first, a hyperelastic formulation, is the model used for all of the simulations shown in the Results section. The second, a hypoelastic formulation, is used to extend the hybrid technique for geometries like inclined chute flow. The notation used mostly follows that of Gurtin, Fried and Anand \cite{Gurtin10}. Bold face Greek and Latin characters are tensors and vectors (differentiated by context), non-bold face Greek and Latin characters are scalers, the trace of a tensor $\bm{A}$ is $tr\bm{A}$, and the deviator, or trace-less part of a tensor is denoted by a subscript 0, such that $\bm{A}_0=\bm{A}-\frac{1}{dim}tr{\bm{A}}\bm{I}$, where $dim$ is the spatial dimension of the domain being considered.

\section{Hyperleastic-Plastic Model} \label{hyperelastic_model}
To begin the discussion of the hyperelastic model, the evolution of the system is governed by the conservation of momentum
\begin{align}
\rho \frac{D \bm{v}}{D t} &= \nabla \cdot \sigma + \rho \bm{f}_{ext} ,
\end{align}
and the conservation of mass
\begin{align}
\frac{D \rho}{D t} + \rho \nabla \cdot \bm{v} = 0 ,
\end{align}
where $\sigma$ is the Cauchy stress tensor, $\frac{D}{D t}$ is the material derivative, and $\bm{f}_{ext}$ denotes any external body forces, like gravity. 

In order to fully close the model, one must further define the kinematics and the constitutive law of the system. In regards to the kinematics, we make use of the classic large deformation kinematics governed by $\bm{F}$, the deformation gradient, defined as
\begin{equation}
\bm{F}=\partial\bm{x}/\partial\bm{X}
\end{equation}
where $\bm{x}$ is the coordinate space of a deformed body and $\bm{X}$ is the coordinate space of the reference body. Furthermore, with the use of plasticity in our constitutive law, we take the Kroner, or multiplicative, decomposition of the deformation gradient, $\bm{F}=\bm{F}^e\bm{F}^p$. $\bm{F}^e$ is the elastic, and $\bm{F}^p$ is the plastic component of the deformation gradient. Again connecting to the constitutive law, we define and make use of $\bm{B}$, the left Cauchy-Green strain tensor defined as $\bm{B}=\bm{F}\bm{F}^T$. 

Being a hyperleastic model, the constitutive law is derived from a strain energy density function.
Under small strains, we desire to have the material behave elastically. In 2D the strain energy density function we use to achieve this is:
\begin{equation}
W = \frac{\kappa}{2} \left[ \frac{1}{2} ( J^2 - 1 ) - \ln J \right] + \frac{1}{2} G ( Tr[ \bar{\bm{B}}^e ] - 2 )
\end{equation}
and in 3D is:
\begin{equation}
W = \frac{1}{2} \kappa \left[ \frac{1}{2} ( J^2 - 1 ) - \ln J \right] + \frac{1}{2} G ( Tr[ \bar{\bm{B}}^e ] - 3 )
\end{equation}
where $\bar{\bm{B}}^e = det\left( \bm{B}^e \right)^{-1/{dim}} \bm{B}^e$ is the isovolumetric component of the elastic left Cauchy-Green strain, $J = det\left( \bm{F} \right)$, and $dim$ is the spatial dimension (2 for 2D and 3 for 3D). The resulting Kirchoff stress is:
\begin{equation}
\tau = \frac{\kappa}{2} \left( J^2 - 1 \right)\bm{I} + G dev[\bar{\bm{b}}^e] \label{kirchoff_stress_hyper}
\end{equation}
where $\kappa$ and $G$ are the bulk and shear moduli respectively of the material.

One of the key properties of a granular media is that like a solid it can resist compression, but under tension it changes phases, acts like a fluid, and cannot resist extension. Therefore to allow the granular medium to separate with no resistance, we consider a "no-tension" rule. We model this effect with a free-flow mode, similar to Dunatunga et al. \cite{Dunatunga:2015:Continuum}. If $det[\bm{B}^e]>1$, the material is in extension. When this occurs, the deformation of the material is projected into the isovolumetric space where $det[\bm{B}^e]=1$, by multiplying by the inverse of $J^2$, or $\bm{B}^e=det\left( \bm{B}^e \right)^{-1/{dim}} \bm{B}^e$. With no elastic volumetric strain under this projection in tension, the pressure drops to 0.

In order to incorporate plasticity, we must define a yield condition and a plastic flow rate. For the yield condition we take a simple Drucker-Prager model defined as
\begin{align}
\Phi = \bar{\tau} - \mu p \leq 0,
\end{align}
where $\mu$ is the internal friction coefficient, $\bar{\tau}$ is the equivalent shear stress and p is the pressure defined by
\begin{equation}
\bar{\tau}=\sqrt{\frac{1}{2}(\bm{\sigma_0}:\bm{\sigma_0})}\label{tau_bar}
\end{equation}
\begin{equation}
p=-\frac{1}{3}\bm{\sigma}\label{pressure_stress}
\end{equation}
Substituting in the elastic constitutive law (\ref{kirchoff_stress_hyper}), the yield condition can also be expressed as
\begin{equation}
\Phi = G \|  dev[ \bar{\bm{b}}^e ] \|_F + \mu \frac{\kappa}{2} \left( J^2 - 1 \right) \label{yield_hyper_strain}
\end{equation}
When $\Phi \leq 0$, the material behaves elastically and no plastic evolution occurs, while when $\Phi > 0$, the material plastically flows. The Drucker-Prager yield criterion looks very similar in form to Coulomb friction, and this is not a coincidence. An analogy can be drawn between the two, such that when the yield function is less than 0, the shear stress (or tangential force) is not enough to overcome the pressure scaled by the internal friction coefficient (normal force and friction coefficient) and the material deforms elastically (does not slip). When the yield criterion is met, then yield, or slip occurs. Much like how the friction coefficient can be used to control the incline angle at which a block at rest will start slipping, the internal friction coefficient $\mu$ is a tuning parameter that controls how easily the material yields, which macroscopically determines behavior like the repose angle of a pile of collapsed grains.

To complete the system definition we require a plastic flow rule. We choose to ignore any sort of plastic softening or hardening effects to keep the flow rule as simple as possible, and so we model the granular continuum as perfectly plastic. We also assume isochoric plastic flow. One advantage of this perfect plasticity approach is that we do not have to track plastic strains, which in a granular medium can be very large. This would result in numerical issues, as $\bm{F}^p$ can become very large in comparison to $\bm{F}^e$. Thus, we only track the evolution of the elastic strain measure of choice, $\bm{B}^e$ through the evolution rule
\begin{equation}
\dot{\bm{B}}^e = \nabla \bm{v} \bm{B}^e + \bm{B}^e \nabla \bm{v}^T + \mathcal{L}_{\bm{v}} \bm{B}^e
\end{equation}
We take into account the yield condition by taking a predictor-corrector style of approach and follow the return mapping method of Simo \cite{Simo:1998}. First the strains and stresses are assumed to be purely elastic and we update $\bm{B}^e$ via
\begin{equation}
\dot{\bm{B}}^e = \nabla \bm{v} \bm{B}^e + \bm{B}^e \nabla \bm{v}^T
\end{equation}
The new predicted $\bm{B}^e$, which we denote as $\bm{B}^{e,*}$, is then used to calculate a new stress and pressure $\bar{\tau}^*$, and $p^*$ respectively. If $\Phi \leq 0$ then the system is not plastically yielding, and so the predictor $*$ quantities are taken as the final, end-step quantities. If however, $\Phi > 0$ then the yield criterion is violated.

To return the material to a valid state, we keep in mind the isochoric plastic flow requirement as well as the fact that final accepted state, which we denote as $\b_p$ must satisfy the yield condition, such that $\Phi{\b_p}$. We also assume that plastic flow is codirectional with the shear stress. With these assumptions in place, the admissible strain is decomposed into
\begin{equation}
\bm{B}^{e,p}=\lambda_1\bm{I}+\lambda_2 dev[det[\bm{B}^{e,*}]] \label{be_split}
\end{equation}
The goal is to solve for $\lambda_1$ and $\lambda_2$ that satisfies the constraints. We therefore substitute (\ref{be_split}) into (\ref{yield_hyper_strain}) to obtain:
\begin{align}
\begin{aligned}
  \Phi\left( \tau \left( \bm{B}^{e,p} \right) \right) &= \mu \| dev[\bm{B}^{e,p}] \|_F + \alpha \frac{\kappa}{2} ( J^2 - 1 ) \\
                                                             &= \mu \lambda_2 \| dev[\bm{B}^{e,*}] \|_F + \alpha \frac{\kappa}{2} ( J^2 - 1 ) \\
&= 0
\end{aligned}
\end{align}
We use the expansions of the determinants such that in 2D, $det[\bm{I} + \bm{A}] = 1 + det[\bm{A}] + Tr[\bm{A}]$ and in 3D, $det[\bm{I} + \bm{A}] = 1 + det[\bm{A}] + Tr[\bm{A}] + \frac{1}{2}Tr[\bm{A}]^2 - \frac{1}{2}Tr[\bm{A}^2]$, to, after some mathematical manipulation arrive at a system of equations for $\lambda_1$ and $\lambda_2$:
\begin{equation}
\lambda_1 = \sqrt{det[\bm{B}^{e,*}] - \lambda_2^2 det[ dev[\bm{B}^{e,*}]]}
\end{equation}
\begin{equation}
\lambda^3_1 + \lambda^3_2 det[dev[\bm{B}^{e,*}]] - \frac{\lambda_1 \lambda^2_2}{2} \|dev[\bm{B}^{e,*}]\|^2_F - det[\bm{B}^{e,*}] = 0
\end{equation}
Solving for $\lambda_1$ and $\lambda_2$ yields the admissible strain state $\b_p$ at the end of the step.

\section{Hypoelastic-Plastic Model} \label{hypoelastic_model}
In general, hypoelastic models differ from hyperelastic models in that the stress is not obtained from a gradient of a strain energy density function with respect to deformation. The specific hypoelastic granular continuum model we use was developed by Dunatunga and Kamrin \cite{Dunatunga:2015:Continuum}. To start one again begins with momentum balance and mass balance
\begin{equation}
\rho\frac{D\bold{v}}{Dt} = \nabla\cdot\bm{\sigma} + \rho\bold{b} \label{mom_bal}
\end{equation}
\begin{equation}
\frac{D\rho}{Dt} + \rho\nabla\cdot\bold{v}=0\label{mass_bal}
\end{equation}
with all terms similarly defined as in the hyperelastic model. A useful quantity, the spatial velocity gradient $\bold{L}$, is defined as
\begin{equation}
\bold{L}=\nabla\bold{v} \label{L_def}
\end{equation}
$\bold{L}$ can be decomposed into a symmetric part (known as the strain rate tensor) and skew part (known as the spin tensor), $\bold{D}$ and $\bold{W}$ respectively, such that
\begin{subequations}
\begin{align}
\bold{L}&=\bold{D}+\bold{W} \label{L_split} \\
\bold{D}&=\frac{1}{2}(\bold{L}-\bold{L}^T) \label{D_def} \\
\bold{W}&=\frac{1}{2}(\bold{L}+\bold{L}^T) \label{W_def}
\end{align}
\end{subequations}
In contrast to the previously described hyperelastic model, here we take additive split of the strain and strain rate-like terms into an elastic and plastic part. For example,
$$\bold{L}=\bold{L}^e+\bold{L}^p$$
The elastic and plastic spatial velocity gradients can then be decomposed into spin and strain rate tensors
$$\bold{L}^e=\bold{D}^e+\bold{W}^e$$
$$\bold{L}^p=\bold{D}^p+\bold{W}^p$$
Due to the fact that a hypoelastic-plastic  model is used and there is no tracking of the deformation gradient, an objective rate must be used to update the stress. While many exist, the Jaumann rate is used here as suggested by Dunatunga and Kamrin, and is defined as
\begin{equation}
\stackrel{\triangle}{\bm{\sigma}}=\dot{\bm{\sigma}}-\bm{W}\cdot\bm{\sigma}+\bm{\sigma}\cdot\bm{W}\label{Jaumann_rate}
\end{equation}
With the basic kinematic variables needed now defined, the next step is defining the constitutive model. As stated in the beginning of this section, this is a hypoelastic-plastic model, and so the elastic constitutive model, plastic yield condition, and plastic flow rule are needed to close the system. The material is assumed to be isotropic and linearly elastic, and the stress is assumed to only be a function of elastic strains. In general the stress rate can then be expressed as a function of the elastic strains contracted with a fourth-order elastic tensor $\mathbb{C}$, or $\stackrel{\triangle}{\bm{\sigma}}=\mathbb{C}:\bm{D}^e$. With the assumptions of isotropocity and first-order linear elasticity, the stress rate can then be more specifically defined as
\begin{equation}
\stackrel{\triangle}{\bm{\sigma}}=2G\bm{D}^e+\lambda tr(\bm{D}^e)\bm{I}\label{constitutive}
\end{equation}
where $G$ is the shear modulus (or second Lam\'e constant) and $\lambda$ is the first Lam\'e constant. However, there is an additional condition on the pressure, which is that
\begin{equation}
	p=
\begin{cases}
	0,							 & \text{if } \rho \leq \rho_c \\
	\frac{K_c}{\rho}(\rho-\rho_c),& \text{if } \rho \geq \rho_c
\end{cases}
\label{pressure_state}
\end{equation}
In other words, if the density of the granular material falls below a certain level, the continuum represents a region of grains that is very loosely packed and has no contacts, and thus cannot support stress. In the physical sense, the grains in this region have entered a gaseous regime (though with no pressure from collisions with the boundary).

The yield condition is very similar in form to the  Drucker-Prager yield condition used in the hyperelastic update, i.e.
\begin{equation}
\bar{\tau} \leq \mu p \label{yield_condition}
\end{equation}
where $\bar{\tau}$ is the equivalent shear stress and p is the pressure, again defined as in (\ref{tau_bar}) and (\ref{pressure_stress}) respectively.
A key difference between the previously explained hyperelastic model and the current hypoelastic model is that the hypoelastic model used by Dunatunga, and subsequently used here, is the introduction of the $\mu(I)$ rheology proposed by Jop et al \cite{Jop:2006:Constitutive}. The $\mu(I)$ rheology proposes a characteristic nondimensional number $I$, defined as
\begin{equation}
I=\dot{\bar{\gamma}}^p\frac{\sqrt{d^2\rho_s}}{\sqrt{p}} \label{inertial_number}
\end{equation}
which gives a measure of the inertia in a sheared granular system relative to the pressure of the system. An empirical fit between $\mu$ and $I$ is given as
\begin{equation}
\begin{cases}
	\mu=\mu(I)=\mu_s+\frac{\mu_2-\mu_s}{I_0/I+1}, & \text{if } I>0 \\
	\mu\leq \mu_s								 & \text{if } I=0
\end{cases}
\label{mu_of_I}
\end{equation}
where $\mu_s$, $\mu_2$ and $I_0$ are material parameters. As suggested by \ref{mu_of_I}, $\mu_s$ is a static friction coefficient, or the value of friction in the limit that $I$ approaches 0. As $I$ approaches infinity $\mu$ approaches $\mu_2$. Though the existence of an asymptotic $\mu_2$ is still debated in literature, it serves as a good approximation for the levels of $I$ reached in the simulations run in this study. Thus the plastic yield condition utilized here is more exactly stated as
\begin{equation}
\bar{\tau} \leq \mu(I) p \label{yield_condition_mu_of_I}
\end{equation}

At plastic yielding, a flow rule must be defined to evolve the plastic strain. A commonly taken assumption that is also taken here is one of spin-less plastic flow, so that $\bold{W}^p=\bold{0}$ and $\bold{L}^p=\bold{D}^p$. Plastic flow codirectionality with the stress deviator and isochoric plastic flow are also taken as assumptions, leading to an plastic flow rate of
\begin{equation}
\bold{L}^p=\bold{\hat{D}}^p(\bm{\sigma})=\frac{1}{\sqrt{2}}\dot{\bar{\gamma}}^p(\bm{\sigma})\frac{\bm{\sigma}_0}{||\bm{\sigma}_0||}\label{flow_rule}
\end{equation}
where $\dot{\bar{\gamma}}^p$ is the equivalent plastic shear strain rate.

As a final note, there is again the desired behavior of a "no tension" rule, in that granular media can not support tensile stress states. While this is partly captured by the pressure dependence on the material density relative to a critical density expressed in \ref{pressure_state}, another check must be done. In the constitutive update to evolve the stress, if it is determined that the pressure of the material is negative (i.e. the material wants to contract in on itself because of volumetric tensile stresses), then the stress is set to 0. Exact implementation details of the stress update can be found in Dunatunga et al, with the relevant density and pressure checks of that update being most relevant for hybridization purposes. 

\section{Material Point Method}
In order to discretize and solve the equations defined in the previous sections, an appropriate method must be chosen. Classically the finite element method has been the method of choice for problems involving solid mechanics. As stated before however, a singular granular system, i.e. flow in an hourglass, has that granular system existing in multiple states at once: a solid bottom pile, a flowing regime down the top of the pile and at the top flowing into the hourglass neck, and a gaseous regime as it exits the neck. Using a method like the finite element to track the deformation of the granular continuum would be nearly impossible, due to the large amounts of non-affine strain that accumulate in the system causing mesh inversions. Remeshing, or a method like the Arbitrary-Lagrangian-Eularian method, could at first glance help resolve this. However the amount of remeshing that needs to occur would incur both a computational penalty for the remeshing algorithm, but also an accuracy penalty due to the need to constantly interpolate quantities.

On the other hand, methods used to solve equations in an Eularian frame for fluid mechanics, like the finite volume method, may then seem appealing. Finite volume however brings with it its own drawbacks in the context of granular media. Finite volume methods have trouble modeling purely solid regimes \textcolor{red}{FINDREFERENCE}. They also do not inherently track free surfaces like Lagrangian finite element would. This free surface tracking is crucial in the problems of interest in granular media study, as the evolution of the free surface, and the interactions of the free surface with surrounding matter, are what ultimately matter in, for example studying the effects of a landslide on anything downhill of the flow zone. Breakaway of granular material from an initial agglomeration of material and the ability to divide that agglomeration into smaller bodies of granular material are also behaviors that are exhibited that cannot be easily captured by finite volume.

The ideal method then is Lagrangian, can track free surfaces, can also handle the large non-affine strains introduced in the liquid and gaseous regimes of granular flow. A class of methods, called particle methods, aim to solve this niche of problem by tracking the evolution of the system through particles, instead of with a mesh. Many types of course exist, including the popular smoothed-particle hydrodynamics (SPH) diffusive element method, and the reproducing kernal method (RKPM) \textcolor{red}{FINDREFERENCE}. All vary in their exact discretization of continuum quantities, representation of connectivity between points, and other details. The continuum discretization method used in this study, known as the Material Point Method (MPM), is a framework that both provides familiarity with methods like the finite element method while adding on the abilities desired.

MPM was developed in the mid 1990s by Sulsky et al and has enjoyed much use and development since \cite{Sulsky:1994}\textcolor{red}{FINDREFERENCE}. MPM is what is known as a mesh-free method, which as the name implies, denotes that there is no permanent mesh used to track deformation. This lack of a permanent mesh of course avoids the mesh deformation issue entirely. As a brief history aside, MPM is a derivative of the fluid-implicit-particle method (FLIP), which is itself a derivative of the particle-in-cell (PIC) method, where PIC was developed in the context of building a method to solve for fluid flow in a Lagrangian frame. Properties of both methods explicitly arise in MPM, which will be discussed later.

\subsection{MPM Algorithm Overview}

\begin{figure}[htp] 
    \centering
    \includegraphics[width=\textwidth]{figs/mpm_scheme_compress_shear_less_pts.pdf}
    \caption{Schematic of a single timestep in MPM.}
    \label{MPM_diagram}
\end{figure}

Figure \ref{MPM_diagram} gives a pictorial overview of a step of the MPM algorithm. In MPM, a continuum body is first discretized (the light red body) via Lagrangian markers (red squares), known as MPM points, as shown in the first two components of Figure \ref{MPM_diagram}. Quantities of interest, like mass, momentum, stress, and any internal variables, are held on these points. It should be noted that there is no explicit notion of connectivity stored on the points between pairs or groups of points, and so no nearest-neighbor search must be conducted, like in SPH or many other particle methods. A temporary (with an emphasis on the "temporary", as the introduction of a mesh may seem contradictory to MPM being classified a mesh-free method) background grid is then introduced as a "computational scratch-pad" . The aforementioned quantities of interest are then projected onto the background grid with a chosen set of basis functions. As a note, while there is no strict requirement on the discretization of the background grid, often a simple Cartesian grid is chosen for convenience. With these quantities now having a nodal representation on the grid (orange circles at the grid nodes), a finite element-like update is conducted. The updated nodal quantities are then projected back onto the MPM points, so that the points are now in an updated state. The background grid is then destroyed, so that no accumulation of strain occurs. With new point quantities, the points are then advected from their old positions (dark red shadows) to their new positions (red markers), completing a timestep of MPM.

\subsection{MPM Formulation and Discretization}
As shown schematically in the previous section, at the beginning of a timestep $n$, each MPM point $p$ has stored on it its position ${\bm{x}_p}^n$, velocity ${\bm{v}_p}^n$, mass ${m_p}^n$, velocity gradient ${\bm{L}_p}^n$, Cauchy stress ${\bm{\sigma}_p}^n$, volume ${V_p}^n$, and for the hyperelastic case, ${{\bm{B}_p}^e}^n$ and ${J_p}^n$. The grid projection of any point quantity $\bm{\phi}_p$ onto a node i is done via the operation
\begin{equation}
\phi_i=\sum_pS_{ip}{\bm{\phi}}_p\label{projection_value}
\end{equation}
where $S_{ip}$ is the value of the basis function $S_i$ at location $\bm{x}_p$, or $S_{ip}=S_i{\bm{x}_p}$. Likewise the grid projection of the gradient of any point quantity $\bm{\phi}_p$ onto a node i is done via the operation
\begin{equation}
\nabla \phi_i=\sum_p \nabla S_{ip}{\bm{\phi}}_p \label{projection_gradient}
\end{equation}
While one is free to choose from any number of function spaces for the basis functions, two types are used in this study. The first are classic linear "hat" functions, which in 1D are defined as
\begin{equation}
S_i(x)=max[0,(1-\frac{|x_i-x|}{h})]\label{linear_basis}
\end{equation}
where h is the element length. The gradient is then defined as
\begin{equation}
	\nabla S_i(x)=
\begin{cases}
	\frac{sgn(x_i-x)}{h},     & \text{if } |x_i-x| \leq h \\
	0,						 & \text{otherwise}
\end{cases}
\label{linear_basis_gradient}
\end{equation}
The second class of basis functions used are known as GIMP (Generalized Interpolation Material Point) basis functions. GIMP basis functions take into account a finite size for the points (instead of a delta function classically used), and integrate the bases across this point domain. This extended support for the GIMP basis functions result in smoother grid crossings and higher order approximations. First order GIMP basis functions (resulting in 2nd order field approximations) were used, with details being found in \cite{Bardenhagen:2004}.

The product of these basis functions in additional directions in 2D and 3D then form the basis in those dimensions. Note that from now on, all basis function values are taken for the point locations at time $n$, and so for brevity the superscript $n$ is not included for the basis functions $S_{ip}$ and gradients $\nabla S_{ip}$.

\begin{algorithm}
  \caption{${\tt Overall \_ MPM \_ Algorithm}$}
  \begin{algorithmic}[1]
  \State ${\tt projectPointMasses}$
  \State ${\tt projectPointMomentum}$
  \If{$\tt Hypoelastic Model$}
  	\State ${\tt updateVolume}$
	\State ${\tt computeHypoelasticCauchyStress}$
  \ElsIf{$\tt HyperelasticModel$}
  	\State ${\tt updateVolumetricStrain}$
	\State ${\tt computeHyperelasticCauchyStress}$ 
  \EndIf
  \State ${\tt projectForces}$ 
  \State ${\tt updateNodalMomentum}$
  \State ${\tt resolveNodalPlaneCollision}$
  \State ${\tt updateNodalVelocityAndAcceleration}$
  \State ${\tt updatePointVelocityGradient}$
  \If{$\tt Hyperelastic Model$}
  	\State ${\tt elasticPrediction}$
	\State ${\tt plasticCorrection}$
  \EndIf
  \State ${\tt updatePointVelocities}$
  \State ${\tt updatePointPositions}$
  \State ${\tt resolvePointPlaneCollision}$
  \State ${\tt clearGridData}$
  \end{algorithmic}
  \label{alg:MPM_algorithm}
\end{algorithm}


To begin, the point masses and momenta are projected onto the nodes via the operations previously described. 
\begin{equation}
{m_i}^n=\sum_pS_{ip}{m_p}^n,\ ({{\bm{mv}}_i})^n=\sum_pS_{ip}({{\bm{mv}}_p})^n\label{mass_and_mom_projection}
\end{equation}
For the hypoelastic model, the point volumes and stresses are then updated. The volume update is described by
\begin{equation}
{V_p}^{n+1}={V_p}^n exp(\Delta tr\bm{L}^n) \label{volume_update}
\end{equation}
The stress update is then calculated with the constitutive law described in Section \ref{hypoelastic_model}, and the exact numerical implementation can be found in Dunantunga and Kamrin \cite{Dunatunga:2015:Continuum}. In the hyperelastic case, the volumetric strain $J$ and stress are also updated here, as described in \ref{hyperelastic_model}.
The external forces ${\bm{b}_i}^n$ and internal forces ${\bm{f}_i}^n$ (internal forces being derived from the divergence of the just-calculated Cauchy stress) are then projected to the grid.
\begin{equation}
{\bm{b}_i}^n=\sum_pS_{ip}{m_p}^n{\bm{b}_p}^n,\ {\bm{f}_i}^n=\sum_p -V_p\bm{{\sigma}_p}^n \cdot \nabla S_{ip}
\label{force_projection}
\end{equation}
Now, the nodes contain both the current momentum and current forces. The change in nodal momentum is then given by
\begin{equation}
\dot{(\bm{mv})^n_i}=\sum_{i}\bm{F}^n_i=\bm{b}^n_i+\bm{f}^n_i\label{momentum_rate}
\end{equation}
The time integration of the nodal momentums can then be done a number of ways, but here a simple forward euler is used, which produces
\begin{equation}
(\bm{mv})^{n+1}_i=(\bm{mv})^n_i+\Delta t(\bm{b}^n_i+\bm{f}^n_i)
\end{equation}
Nodal interactions with boundaries are then taken into account. In the current code, two types of boundaries are supported: "sticky" rigid planes and "sliding" rigid planes. The algorithm for this interaction check loops through all grid nodes $i$, and checks to see if any mass has been projected to the node. If so, then the code loops through all defined planes, calculating the distance of the grid point to the given plane, to check for a nodal collision with the plane. If there is a collision found, then the relative velocity of the point to the plane is calculated. For a "sliding" boundary, the normal relative momentum is set to zero (allowing movement in the tangential velocity, and hence the "sliding"), while in the "sticky" boundary case the entire nodal momentum $(\bm{mv})^{n+1}_i$ is set to $\bm{0}$.

Next, the new nodal velocities and accelerations are calculated as
\begin{equation}
\bm{v}^{n+1}_i=(\bm{mv})^{n+1}_i/m^n_i
\end{equation}
\begin{equation}
\bm{a}^{n+1}_i=\frac{(\bm{mv})^{n+1}_i-(\bm{mv})^n_i}{\Delta t m^n_i}
\end{equation}
The new nodal velocities are used to calculate the new velocity gradients $\bm{L}^{n+1}_p$ on the points with
\begin{equation}
\bm{L}^{n+1}_p=\sum_{i}\bm{v}^{n+1}_i \otimes\nabla S_{ip}
\end{equation}
In the hyperlastic model, the elastic prediction and plastic correction steps are then conducted, as explained in \ref{hyperelastic_model}.

The next step, the update of the point velocity, is one that deserves extra attention. Two quantities are temporarily introduced, the PIC velocity $\bm{v}_{pic}$ and the point acceleration $\bm{a}_p$, defined as
\begin{equation}
\bm{v}_{pic}=\sum_{i}S_{ip}\bm{v}^{n+1}_i
\label{pic_update}
\end{equation}
\begin{equation}
\bm{a}_p=\sum_{i}S_{ip}\bm{a}^{n+1}_i
\end{equation}
From this it can be seen that there are two possible avenues to update the point velocity to $\bm{v}^{n+1}_p$. One, called the PIC update (so-called because this is the point velocity update that the PIC method used), directly uses the $\bm{v}_{pic}$ velocity as the new point velocity, so $\bm{v}^{n+1}_p=\bm{v}_{pic}$. This means that the point velocities are directly interpolated from the background grid velocities, and thus the velocity field is constrained by the basis functions. The other, called the FLIP update (again so-called because FLIP uses this as its velocity update), instead uses the nodal accelerations to construct a point acceleration. The FLIP velocity, $\bm{v}_{flip}$ is then obtained by
\begin{equation}
\bm{v}_{flip}=\bm{v}^n_p+\Delta t \bm{a}_p
\label{flip_update}
\end{equation}
With this construction, the velocities live in a higher order vector space, allowing for higher order kinematic modes. The macroscopic result of these two updates is that, often, flows with PIC updates have a high degree of dissipation and do not conserve angular momentum. Physically realistic voritical flow and oscillations either do not appear or are quickly damped. This dissipative quality however means that PIC schemes are often stable. On the other hand, FLIP updated flows more often preserve those vortical effects and oscillations and better conserve angular momentum. This though means that instabilities can form and will not be damped out. 

A strategy that is used to obtain some middle ground between the two strategies is to simply take a linear combination of the two updates. This is expressed functionally as
\begin{equation}
\bm{v}^{n+1}_p=(1-\alpha)\bm{v}_{pic} + \alpha \bm{v}_{flip}
\end{equation}
where $\alpha$ is a parameter used to tune how much one wants a PIC vs a FLIP update. A large $\alpha$ value is often used, as better angular momentum conservation is usually desired over better numerical stability, though a small portion of PIC velocity still helps with stability. In this study, $\alpha$ was set to between 0.95 and 1.0 for all simulations. 

The points are then advected via
\begin{equation}
\bm{x}^{n+1}_p=\bm{x}^n_p+\Delta t\sum_{i}S_{ip}\bm{v}^{n+1}_i
\end{equation}
Another collision check is conducted on the advected position of the points. This collision check is very similar to the nodal check, wherein all of the points are looped over, a check for collision against any rigid planes is done, and the normal relative velocity is set to 0 for sliding planes and the entire velocity is set to 0 for sticky planes. Finally, the nodal quantities on the grid are all set to 0, effectively resetting the grid state.

\subsection{Relevant Notes on MPM}
As alluded to with the "FEM-like solve", MPM can be interpreted as a finite element method with a single point quadrature integration rule, where the quadrature points are the MPM points. Both start from a weak form formulation and are discretized into points/elements, with interpolation conducted under some basis. However in MPM, there is no strict notion of connectivity. Instead, communication of points is done through the projection of the point quantities to the nodes.

In the original MPM formulation described by Sulsky, the extent of the points are described via $\delta(x)$ functions, and so only project to the grid nodes of the element that the point is currently in. If two points project to the same grid node, then connectivity is in effect established between the points. This then implies that if two points start out "connected", or are at most in directly adjacent cells, and then become separated by a completely empty element, they lose connectivity, completely separating the two points. While this can be useful in some applications to capture fracture, such as in the modeling of snow breakage in graphics usage \textcolor{red}{FINDREFERENCE}, it should be stated that this in fact numerical error. While schemes that integrate a point across a finite extent, such as GIMP, ameliorate this problem somewhat, this is not a complete fix. Thus while MPM avoids having grid cell inversions and other issues inherent in a completely lagrangian mesh, a sort of numerical fracture becomes the byproduct.

Numerical fracture is not the only issue that can arise however. In a somewhat less catastrophic error, the fact that the MPM points are effectively quadrature points, and they are allowed to advect from cell to cell, means that integration errors can occur. For example, a simulation can begin with the same number of MPM points per cell for all cells representing the body, exactly integrating those elements. Over time however, the volumes that the MPM points represent deform and change in size, and the points themselves move. This can mean that if at a certain timestep, a cell contains much fewer points than the simulation started with, that portion of the body can be under-integrated. On the other hand, a cell may have a large concentration of points, with the volumes that the points represent actually overlapping; over-integration of that portion of the domain then results.

Techniques exist to address these points. To better track the deformation of the volume that the MPM point represents in order to help with over and under integration, the previously mentioned GIMP as well as the Convected Particle Domain Interpolation (CPDI) technique can be used. CPDI, as its name implies, convects and keeps track of the representation of the MPM point volume, so that the area over which one integrates changes with MPM deformation \textcolor{red}{FINDREFERENCE}.

Resampling can also help to avoid over or under integration, keeping the MPM point concentration similar to what the simulation started with. A resampling algorithm was developed by Yue et al in the context of using MPM for foam modeling, where very large deformations occur that can result in under integration \cite{Yue:2015:Continuum}. This technique, called Avoid a Void, is used in the current study both for the MPM simulations as well as the basis for a component of hybridization later discussed. 